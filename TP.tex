\documentclass[10pt,a4paper]{article}

\input{AEDmacros}
\usepackage{caratula} % Version modificada para usar las macros de algo1 de ~> https://github.com/bcardiff/dc-tex


\titulo{Trabajo pr\'actico 1: Especificac\'ion y WP}
\subtitulo{Elecciones Nacionales}

\fecha{\today}

\materia{Algoritmos y Estructuras de Datos}
\grupo{sudo\_rm-rf\_/*}

\integrante{Rocca, Santiago}{152/23}{santiagrocca17@gmail.com}
\integrante{Fisz, Maximiliano}{586/19}{maximilianofisz@gmail.com}   
\integrante{Gomez, Abril}{574/20}{goskema@gmail.com}
\integrante{L\'opez, Gonzalo}{1017/22}{gonzalo.esloga.uba@gmail.com}
% Pongan cuantos integrantes quieran

% Declaramos donde van a estar las figuras
% No es obligatorio, pero suele ser comodo
\graphicspath{{../static/}}

\begin{document}

\maketitle

\section{Especificaci\'on}
	


\subsection{hayBallotage}
		\subsubsection{Main}

			\begin{proc}{hayBallotage}{\In escrutinio : \TLista{\ent}}{\bool}
	
				\requiere{eleccionValida(escrutinio))}
	
				\asegura{res=\neg((cond1HayBallotage(escrutinio))\lor_{ L }(cond2HayBallotage(escrutinio)))}

			\end{proc}

		\subsubsection{Predicados Especificos}

			\pred{cond1HayBallotage}{\In escrutinio : \TLista{\ent}}{(\exists n:\ent)(0 \leq n < |escrutinio| \land_{ L }(porcentajeDeVotos(escrutinio,escrutinio[x])>45)}


			\pred{cond2HayBallotage}{\In escrutinio : \TLista{\ent}}{(\exists n:\ent)(0 \leq n < |escrutinio| \land_{ L }(porcentajeDeVotos(escrutinio,escrutinio[x])>40) \: \land_{ L } \\ \neg(\exists x:\ent)(0\leq x < |escrutinio| \land_{ L }  (\neg(n=x) 	\land_{ L } ((escrutinio[n] - escrutinio[x])>10)) }


			


\subsection{hayFraude}
		\subsubsection{Main}
			
			\begin{proc}{hayFraude}{(\In escrutinio\_Presidencial: \TLista{\ent}, \In escrutinio\_senadores: \TLista{\ent}⟩, \In escrutinio\_diputados: \TLista{\ent}}{\bool}
			%UMBAL ELECTORAL NO SABEMOS BIEN
				\requiere{umbralElectoral(escrutinio\_senadores)\: \land_{ L } \:eleccionValida(escrutinio\_Presidencial) \: \land_{ L } \:\\ eleccionValida(escrutinio\_senadores) \: \land_{ L } \: eleccionValida(escrutinio\_diputados)}
				
				\asegura{res=\neg(((|sumaDeVotos(escrutinio\_Presidencial) = sumaDeVotos(escrutinio\_Senadores)) \land_{L} \\ (sumaDeVotos(escrutinio\_Presidencial) = sumaDeVotos(escrutinio\_Diputados)))}
			
			\end{proc}






		
	\subsection{obtenerSenadoresEnProvincia}
		\subsubsection{Main}
			
			\begin{proc}{obtenerSenadoresEnProvincia}{\In escrutinio : \TLista{\ent}}{\ent \texttimes \ent}

                			\requiere{eleccionValida(escrutinio) \land minimoDePartidos(escrutinio)}
                			\asegura{(\exists!x : \ent) (0 \leq x < |escrutinio| - 1) \land_{L} ((\exists!y : \ent) (0 \leq y < |escrutinio| - 1) \land_{L} ((\forall i : \ent)(0 \leq i < |escrutinio| \land \neg (i= x) \land  \neg(i = y)) \longrightarrow_{L } escrutinio[i] < escrutinio[res_y] < escrutinio[res_x]))}

            	\end{proc}
		
		\subsubsection{Predicados Especificos}
			\pred{minimoDePartidos}{\In escrutinio: \TLista{\ent}}{|escrutinio| \geq 3}

	


	 \subsection{calcularDHondtEnProvincia}
		\subsubsection{Main}
			\begin{proc}{calcularDHondtEnProvincia}{\In cant\_bancas: \ent, \In escrutinio: \TLista{\ent} } {\TLista{\TLista{\ent}}}
				 
				\requiere{eleccionValida(escrutinio) \land umbralElectoral(escrutinio) \land cant\_bancas>0)}
				\asegura{((\forall \: x:\ent)(0\leq n < cant\_bancas) \land_{L} (\forall \: x:\ent)(0\leq n <|escrutinio|))\longrightarrow_{L } \: (res[x][n]=\frac{escrutinio[x]}{n+1})}

			\end{proc}



	\subsection{obtenerDiputadosEnProvincia}
		\subsubsection{Main}
			\begin{proc}{obtenerDiputadosEnProvincia}{\In cant\_bancas: \ent, \In escrutinio: \TLista{\ent}, \In dHondt: \TLista{\TLista{\ent}}} {\TLista{\ent}}
				 
				\requiere{eleccionValida(escrutinio) \land umbralElectoral(escrutinio)}
				\asegura{(\forall \: r:\ent)(0\leq r < |escrutinio| - 1 \longrightarrow_{L} res[r] = bancasDe(r, cant\_bancas, dHondt))}
			\end{proc}

		\subsubsection{Predicados Especificos}
	
			\pred{cocienteGanador}{\In indicePartido: \ent, \In bancaEnDisputa: \ent, \In dHont: \TLista{\TLista{\ent}}}{res = True \longleftrightarrow
(\forall \: i: \ent)(0 \leq i < |dHont| -1 \land \neg(i = indicePartido) \longrightarrow_{L } dHont[bancaEnDisputa][indicePartido] > dHont[bancaEnDisputa][i])}






	\subsection{validarListasDiputadosEnProvincia}
		\subsubsection{Main}
			
			\begin{proc}{(\In cant\_bancas: \ent, \In listas: \TLista{\TLista{dni:\ent \times genero: \ent}}}{\bool}
			
				\requiere{(cant\_bancas>0) \: \land (dni>0) \: \land \:  (1\leq genero \geq 2)}
				\asegura{(\forall \: \: x:\ent)(0\leq x<|listas|) \longrightarrow_{L} (cantCandidatosCorrecta(cant\_bancas,\: listas[x])\land_{L} altGenero(listas[x])}			

			\end{proc}
		
		\subsubsection{Predicados Especificos}
		
			\pred{cantCandidatosCorrecta}{cant\_bancas: \ent, partido: \TLista{dni:\ent \times genero: \ent}}{cant\_bancas=|partido|}

			\pred{altGenero}{partido: \TLista{dni:\ent \times genero: \ent}}{(((\forall n: \ent)(n>0))\longrightarrow_{L} ( ((n\mod2=0)\longrightarrow_{L} (partido[n,1]=1)) \land_{L } ((n\mod2=1)\longrightarrow_{L} (partido[n,1]=2)) \lor_{L} ((n \mod 2 = 0) \longrightarrow_{L} (partido[n,1]=2 \land_{L} (n \mod 2=1) \longrightarrow_{L} partido[n,1]=1))}
	





	\subsection{Auxiliares}

		\aux{sumaDeVotos}{\In escrutinio : \TLista{\ent}}{\ent}{\sum\limits_{i=0}^{|escrutinio| - 1} escrutinio[i]}

		\aux{porcentajeDeVotos}{\In escrutinio: \TLista{\ent}, \In partido: \TLista{\ent}}{\float}{sumaDeVotos(escrutinio)^{-1} *escrutinio[partido]*10^{2})}

		\aux{bancasDe}{\In indicePartido: \ent, \In bancas, \In dHont  \TLista{\TLista{\ent}}}{\ent}{ \\ \sum\limits_{p=0}^{bancas - 1} if \: cocienteGanador(indicePartido, p, dHont) \: then \: 1 \:  else \: 0}
	





	\subsection{Predicados Universales}
		
		\pred{noHayRepetidos}{\In escrutinio : \TLista{\ent}}{(\forall x: \ent)(0 \leq x < |escrutinio| \: \longrightarrow_{L} \: ((\forall y: \ent)(0 \leq y < |escrutinio| \: \land  \: \neg(x=y) \: \longrightarrow_{L }\neg(escrutinio[x] = escrutinio[y])))}
		
		\pred{cantVotosValidos}{\In escrutinio : \TLista{\ent}}{((\forall x: \ent)(0 \leq x < |escrutinio|) \: \longrightarrow_{L } (escrutinio[x] \geq 0))}
	
		\pred{escrutinioValdio}{\In escrutinio: \TLista{\ent}}{|escrutinio| \geq 2}
		
		\pred{EleccionValida}{\In escrutinio: \TLista{\ent}}{nohayRepetidos(escrutinio) \land cantVotosValidos(escrutinio) \land escrutinioValido(escrutinio)}
		
		\pred{umbralElectoral}{\In escrutinioSen : \TLista{\ent}}{((\forall x: \ent)(0 \leq x < |escrutinio|) \: \longrightarrow_{L } (escrutinioSen[x] > 3))}
		


\section{Implementaciones y demostraciones de correctitud}

	\subsection{Implementaciones}
		
		\subsubsection{hayBallotage}
		% Para hacer que quede todo en una misma linea, se puede usar minipage
			\begin{minipage}[t]{\textwidth}
				\begin{lstlisting}[caption={()},label=code:for]
					res:=True
					tans:=0
					primero:=0
					segundo:=0
					i:=0
					suma :=0
					while (escrutinio.size() > i) do
						suma:= suma + escrutinio[i]
						i:=i+1
					endwhile
					i:=0
					while (escrutinio.size() > i) do
						escrutinio[i]=(escrtuinio[i]*100)/suma
						i:=i+1
					endwhile
					i:=0
					while (escrutinio.size() > i) do 
						if (segundo < escrutinio[i]) then
							segundo:=escrtuinio[i]
						else:
							skip
						endif
						if (primero < segundo) then
							trans := primero
							primero := segundo
							segundo := trans
						else:
							skip
						endif
						i:=i+1
					endwhile
					if (primero > 45) then
						res := false
					else
						if ((primero > 40) && (primero - segundo >= 10)) then
							res:=false
						else
							skip
						endif
					endif

				\end{lstlisting}
			\end{minipage}

		\subsubsection{hayFraude}
			\begin{minipage}[t]{\textwidth}
				\begin{lstlisting}[caption={()},label=code:for]
						res := True
						sumaPres:=0
						sumaDip:=0
						sumaSen:=0				
						while (escrutinio_Presidencial.size() > i) do
							sumaPres:= sumaPres + escrutinio_Presidencial[i]
							i:=i+1
						endwhile
						i:=0
						while (escrutinio_Diputados.size() > i) do
							sumaDip:= sumaDip + escrutinio_Diputados[i]
							i:=i+1
						endwhile
						i:=0
						while (escrutinio_Senadores.size() > i) do
							sumaSen:= sumaSen + escrutinio_Senadores[i]
							i:=i+1
						endwhile
						if (sumaPres = sumaDip && sumaPres = sumaSen) then 
							res := False
						else:
							skip
						endif
				\end{lstlisting}
			\end{minipage}

		\subsubsection{obetenerSenadores}
			\begin{minipage}[t]{\textwidth}
				\begin{lstlisting}[caption={()},label=code:for]
						tran := 0
						primero := 0
						segundo :=0
						i := 0
						while (escrutinio.size() > i) do
							if (escrutinio[segundo] < escrutinio[i]) then
								segundo := i
							else:
								skip
							endif
							if (escrutinio[primero] < escrutinio[segundo]) do
								trans := primero
								primero := segundo
								segundo :=  trans
							else:
								skip
							endif
							i := i + 1
						endwhile	
						res := (primero,segundo)
				\end{lstlisting}
			\end{minipage}
	
		\subsubsection{obetenerSenadores}
			\begin{minipage}[t]{\textwidth}
				\begin{lstlisting}[caption={()},label=code:for]
					res1:=True
				
					i :=0
					while (listas.size() - 1 > i) do
						lf (listas[i].size() != listas[i+1].size()) then	
							res:=false
						else:
							skip
						endif
						i := i+1
					endhwihile
					if (listas[0].size() != cant_bancas) then
						res := false
					else:
						skip
					endif
					i := 0
					j := 0
					while (listas.size() > i) do
						aux := listas[i][0][1]
						while (listas[i].size() - 1 > j) do
							if (listas[i][j][1] == listas[i][0][1] && listas[i][j+1][1] != listas[i][0][1] then
								res:=false
							else:
								skip
							endif
							j := j + 2
						endwhile
						i := i + 1
					endwhile
				%%%%%%%%%%%%%%%PROBLEMAS CON CANDIDATOS IMPARES
				\end{lstlisting}
			\end{minipage}
		
		



Lo principal: las fórmulas. Se puede poner en una linea, como $x_i = x_{i-1} + x_{i-2}$, o ponerse más grande:

\begin{equation}
	\sum\limits_{i=0}^{n} i
	\label{eq:1}
\end{equation}

Y se pueden citar ecuaciones con \verb|\eqref{nombreDeEq}|: \eqref{eq:1}

Ejemplo de itemizado:

\begin{itemize}
	\item Item 1
	\item Item 2
	\item Item 3
\end{itemize}

Ejemplo de enumerado con menor distancia entre items:

\begin{enumerate} \setlength\itemsep{0cm}
	\item Item 1
	\item Item 2
	\item Item 3
\end{enumerate}

Podemos escribir mucho texto. Mucho texto. Mucho texto. Mucho texto. Mucho texto. Mucho texto. Mucho texto. Mucho texto. Mucho texto. Mucho texto. Mucho texto.

Otro párrafo. Otro párrafo. Otro párrafo. Otro párrafo. Otro párrafo. Otro párrafo. Otro párrafo. Otro párrafo. Otro párrafo. Otro párrafo. Otro párrafo. Otro párrafo. Otro párrafo.

\vspace{0.3cm}

Le agregamos una separación entre párrafos. Le agregamos una separación entre párrafos. Le agregamos una separación entre párrafos. Le agregamos una separación entre párrafos. Le agregamos una separación entre párrafos.

\vspace{0.3cm}

La tabla \ref{tab:ejemplo} es un ejemplo de cómo se hace una tabla.

\begin{table}[h!]
	\centering
	\begin{tabular}{||l c c r||} 
		\hline
		Col1 & Col2 & Col2 & Col3 \\ [0.5ex] 
		\hline\hline
		1 & 6 & 87837 & 787 \\ 
		2 & 7 & 78 & 5415 \\
		3 & 545 & 778 & 7507 \\
		4 & 545 & 18744 & 7560 \\
		5 & 88 & 788 & 6344 \\
		\hline
	\end{tabular}
	\caption{Ejemplo de tabla}
	\label{tab:ejemplo}
\end{table}


La figura \ref{fig:subfigs} es un ejemplo de cómo se agrega una imagen.

\begin{figure}[ht]
	\centering
	\includegraphics[width=0.6\textwidth]{logo_dc.jpg}
	\caption{Ejemplo de figura}
	\label{fig:ejemplo}
\end{figure}

\begin{figure}[ht!]
	\begin{subfigure}{0.5\textwidth}
		\includegraphics[width=0.9\linewidth]{LaTeX-project} 
		\caption{Logo de LaTeX}
		\label{fig:subfig1}
	\end{subfigure}
	\begin{subfigure}{0.5\textwidth}
		\includegraphics[width=0.7\linewidth]{TeX}
		\caption{Logo de TeX}
		\label{fig:subfig2}
	\end{subfigure}
	\caption{Ejemplo para poner dos figuras juntas. Y citarlas por separado a (\subref{fig:subfig1}) y (\subref{fig:subfig2}).}
	% OJO: el caption siempre va antes del label
	\label{fig:subfigs}
\end{figure}



% Para hacer que quede todo en una misma linea, se puede usar minipage
%\begin{minipage}[t]{\textwidth}
	\begin{lstlisting}[caption={Ejemplo de código (usando los estilos de la cátedra, ver las macros para más detalles)},label=code:for]
res := 0;
i := 0;
while (i < s.size()) do
	res := res + s[i];
	i := i + 1
endwhile
	\end{lstlisting}
%\end{minipage}

Si se pone un label al \verb|lstlisting|, se puede referenciar: Código \ref{code:for}.


\subsection{Macros de la cátedra para especificar}

\begin{proc}{nombre}{\In paramIn : \nat, \Inout paramInout : \TLista{\ent}}{tipoRes}
	%    \modifica{parametro1, parametro2,..}
	\requiere{expresionBooleana1}
	\asegura{expresionBooleana2}
	\aux{auxiliar1}{parametros}{tipoRes}{expresion}
	\pred{pred1}{parametros}{expresion} 
\end{proc}

\aux{auxiliarSuelto}{parametros}{tipoRes}{expresion}
% \paraTodo{variable}{tipo}{expresion}
% \existe{variable}{tipo}{expresion}
% Pueden tener [unalinea] para que no se divida en varias lineas
\pred{predSuelto}{parametros}{\paraTodo[unalinea]{variable}{tipo}{algo \implicaLuego expresion}}
\pred{predSuelto}{parametros}{\existe[unalinea]{variable}{tipo}{algo \yLuego expresion}}








\end{document}
