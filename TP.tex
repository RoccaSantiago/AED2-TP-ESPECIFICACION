\documentclass[10pt,a4paper]{article}

\input{AEDmacros}
\usepackage{caratula} % Version modificada para usar las macros de algo1 de ~> https://github.com/bcardiff/dc-tex
\usepackage{xcolor}

\titulo{Trabajo pr\'actico 1: Especificac\'ion y WP}
\subtitulo{Elecciones Nacionales}

\fecha{\today}

\materia{Algoritmos y Estructuras de Datos}
\grupo{sudo\_rm-rf\_/*}

\integrante{Rocca, Santiago}{152/23}{santiagrocca17@gmail.com}
\integrante{Fisz, Maximiliano}{586/19}{maximilianofisz@gmail.com}   
\integrante{Gomez, Abril}{574/20}{goskema@gmail.com}
\integrante{L\'opez, Gonzalo}{1017/22}{gonzalo.esloga.uba@gmail.com}
% Pongan cuantos integrantes quieran

% Declaramos donde van a estar las figuras
% No es obligatorio, pero suele ser comodo
\graphicspath{{../static/}}

\begin{document}

\maketitle

\section{Especificaci\'on}
	

    \subsection{General}


	\subsubsection{Predicados Universales}
		
		\pred{noHayRepetidos}{\In escrutinio : \TLista{\ent}}{(\forall x: \ent)(0 \leq x < |escrutinio| \: \longrightarrow_{L} \: ((\forall y: \ent)(0 \leq y < |escrutinio| \: \land  \: \neg(x=y) \: \longrightarrow_{L }\neg(escrutinio[x] = escrutinio[y])))}
		
		\pred{cantVotosValidos}{\In escrutinio : \TLista{\ent}}{((\forall x: \ent)(0 \leq x < |escrutinio|) \: \longrightarrow_{L } (escrutinio[x] \geq 0))}
	
		\pred{escrutinioValdio}{\In escrutinio: \TLista{\ent}}{|escrutinio| \geq 2}
		
		\pred{EleccionValida}{\In escrutinio: \TLista{\ent}}{nohayRepetidos(escrutinio) \land cantVotosValidos(escrutinio) \land escrutinioValido(escrutinio)}
		
		\pred{umbralElectoral}{\In escrutinioDip : \TLista{\ent}}{((\exists x: \ent)(0 \leq x < |escrutinioDip|) \: \land_{L} (escrutinioDip[x] > (|escrutinioDip| * 3) / 100))}

        \pred{minimoDePartidos}{\In escrutinio: \TLista{\ent}}{|escrutinio| \geq 3}


    \subsubsection{Auxiliares}

		\aux{sumaDeVotos}{\In escrutinio : \TLista{\ent}}{\ent}{\sum\limits_{i=0}^{|escrutinio| - 1} escrutinio[i]}

		\aux{porcentajeDeVotos}{\In escrutinio: \TLista{\ent}, \In votosPartido: \ent}{\float}{sumaDeVotos(escrutinio)^{-1} \; * \; votosPartido \; * \; 10^{2})}

		\aux{bancasDe}{\In indicePartido: \ent, \In bancas, \In dHondt  \TLista{\TLista{\ent}}}{\ent}{ \\ \sum\limits_{p=0}^{bancas - 1} if \: cocienteGanador(indicePartido, p, dHondt) \: then \: 1 \:  else \: 0}

        \aux{bancasGanadas}{\In dH: \TLista{\TLista{\ent}},\In max: \ent, \In r: \ent}{\ent}{\\ \sum_{j=0}^{dH[r]-1} if \; max>MaxAnteriores(dH, dH[r][j]) \; then \; 1 \; else \; 0}

        \aux{MaxAnteriores}{\In dH: \TLista{\TLista{\ent}},\In max: \ent, \In r: \ent}{\ent}{\sum^{|dH|-1}_{j=0} \sum^{dH[j]-1}_{i=0} if \; dH[j][i]>r \; then \; 1 \; else \; 0}
	


\subsection{hayBallotage}
		\subsubsection{Main}

			\begin{proc}{hayBallotage}{\In escrutinio : \TLista{\ent}}{\bool}
	
				\requiere{eleccionValida(escrutinio))}
	
				\asegura{res = True \longleftrightarrow \neg((partidoMayorA45\%(escrutinio))\lor (partidoMayorA40\%ConDiferencia(escrutinio)))}

			\end{proc}

		\subsubsection{Predicados Especificos}


			\pred{partidoMayorA40\%ConDiferencia}{\In escrutinio : \TLista{\ent}}{(\exists n:\ent)(0 \leq n < |escrutinio| -1 \land_{ L }(porcentajeDeVotos(escrutinio,escrutinio[n])>40) \: \land_{ L } \\ \neg(\forall x:\ent)(0\leq x < |escrutinio|-1 \land  (\neg(n=x) \longrightarrow_{ L } ((escrutinio[n] - escrutinio[x])>10)) }


			


\subsection{hayFraude}
		\subsubsection{Main}
			
			\begin{proc}{hayFraude}{\In escrutinio\_Presidente: \TLista{\ent}, \In escrutinio\_Senadores: \TLista{\ent}, \In escrutinio\_Diputados: \TLista{\ent}}{\bool}
				\requiere{eleccionValida(escrutinio\_Presidente) \: \land\: eleccionValida(escrutinio\_Senadores) \: \land \: \\ eleccionValida(escrutinio\_diputados) \land minimoDePartidos(escrutinio\_Senadores) \land \\ 
(|escrutinio\_Presidente| = |escrutinio\_Senadores| = |escrutinio\_Diputados|)}
				
				\asegura{res=\neg(((sumaDeVotos(escrutinio\_Presidente) = sumaDeVotos(escrutinio\_Senadores)) \land \\ (sumaDeVotos(escrutinio\_Presidente) = sumaDeVotos(escrutinio\_Diputados)))}
			
			\end{proc}






		
	\subsection{obtenerSenadoresEnProvincia}
		\subsubsection{Main}
			
			\begin{proc}{obtenerSenadoresEnProvincia}{\In escrutinio : \TLista{\ent}}{\ent \texttimes \ent}

                			\requiere{eleccionValida(escrutinio) \land minimoDePartidos(escrutinio)}
                			\asegura{(\exists!x : \ent) (0 \leq x < |escrutinio| - 1 \land_{L} ((\exists!y : \ent) (0 \leq y < |escrutinio| - 1 \land_{L} ((\forall i : \ent)(0 \leq i < |escrutinio| -1 \land \neg (i= x) \land  \neg(i = y) \longrightarrow_{L } (escrutinio[i] < escrutinio[y] < escrutinio[x] \land res_0 = x \land res_1 = y ))))}

            	\end{proc}
		
		\subsubsection{Predicados Especificos}
			

	


	 \subsection{calcularDHondtEnProvincia}
		\subsubsection{Main}
			\begin{proc}{calcularDHondtEnProvincia}{\In cant\_bancas: \ent, \In escrutinio: \TLista{\ent} } {\TLista{\TLista{\ent}}}
				 
				\requiere{eleccionValida(escrutinio) \land umbralElectoral(escrutinio) \land cant\_bancas>0)}
				\asegura{((\forall \: i:\ent)(0\leq i < cant\_bancas) \land_{L} (\forall \: j:\ent)(0\leq j <|escrutinio|))\longrightarrow_{L } \\ (res[j][i]=\frac{escrutinio[j]}{i+1} \land \frac{escrutinio[j]}{i+1} \geq 0)}

			\end{proc}



	\subsection{obtenerDiputadosEnProvincia}
		\subsubsection{Main}
			\begin{proc}{obtenerDiputadosEnProvincia}{\In cant\_bancas: \ent, \In escrutinio: \TLista{\ent}, \In dHondt: \TLista{\TLista{\ent}}}{\TLista{\ent}}
				 
				\requiere{eleccionValida(escrutinio) \land umbralElectoral(escrutinio) \land coeficientesDistintos(dHondt) \\ \land esMatriz(dHondt) \land
                matrizDelEscrutinio(dHondt, \: cant\_bancas, \: escrutinio)}				
                    \asegura{(\forall \: r:\ent)(0\leq r < |escrutinio| - 1 \longrightarrow_{L} res[r] = bancasGanadas(dHondt, cant\_bancas, r)) \; \land \; |res| = |dHondt|}
    
	        \end{proc}

         \subsubsection{Predicados Especificos}

        \pred{esMatriz}{\In dH: \TLista{\TLista{\ent}}}{True \longleftrightarrow 
        (\forall i : \ent)(0 \leq i < |dH| - 1 \longrightarrow |dH[i]| = |dH[i+1]|) 
        }

        \pred{matrizDelEscrutinio}{\In dH: \TLista{\TLista{\ent}}, \In cant\_bancas: \ent, \In escrutinio: \TLista{\ent}}{True \longleftrightarrow 
        ((\forall \: i:\ent)(0\leq i < cant\_bancas) \land_{L} (\forall \: j:\ent)(0\leq j <|escrutinio|))\longrightarrow_{L } dH[j][i]=\frac{escrutinio[j]}{i+1}) 
        }
    

   
   
   \pred{coeficientesDistintos}{\In DHondt: \TLista{\TLista{\ent}}}{
   (\forall j:\ent)(0\leq j < |DHondt| \longrightarrow_{L} ((\forall i:\ent)(0\leq i<|DHondt[j]|\longrightarrow_{L} \neg ((\exists z:\ent)(0\leq z <|DHondt| \land  (z\neq j) \land_{L} \\ ((\exists t:\ent)(0\leq t<|DHondt[z]| \land (t\neq i) \land_{L} DHondt[z][t] = DHondt[j][i]))))))
    }
   \pred{todosPositivos}{\In DHondt : \TLista{\TLista{\ent}}}{(\forall j:\ent)(0\leq j <|DHondt|\longrightarrow_{L} ((\forall i:\ent)(0\leq i < |DHondt[j]|\longrightarrow_{L} DHondt[j][i]>0)))}
   
   
   
   






	\subsection{validarListasDiputadosEnProvincia}
		\subsubsection{Main}
			
			\begin{proc}{(\In cant\_bancas: \ent, \In listas: \TLista{\TLista{dni:\ent \times genero: \ent}}}{\bool}
			
				\requiere{(cant\_bancas>0) \: \land (\forall \: x:\ent)
    (0\leq x < |listas| \longrightarrow_{L} listas[x]_{0} > 0 \: \land \: 1 \leq listas[x]_{1} \leq 2)}
				\asegura{(\forall \: \: partido:\ent)(0\leq partido<|listas|) \longrightarrow_{L} (cantCandidatosCorrecta(cant\_bancas,\: listas[partido])\land altGenero(listas[partido])}			

			\end{proc}
		
		\subsubsection{Predicados Especificos}
		
			\pred{cantCandidatosCorrecta}{cant\_bancas: \ent, partido: \TLista{dni:\ent \times genero: \ent}}{cant\_bancas=|partido|}

			\pred{altGenero}{partido: \TLista{dni:\ent \times genero: \ent}}{(((\forall n: \ent)(0 \leq n < |partido|))\longrightarrow_{L} ( ((n\mod2=0)\longrightarrow_{L} (partido[n]_1=1)) \land_{L } ((n\mod2=1)\longrightarrow_{L} (partido[n]_1=2)) \lor_{L} ((n \mod 2 = 0) \longrightarrow_{L} (partido[n]_1=2 \land_{L} (n \mod 2=1) \longrightarrow_{L} partido[n]_1=1))}
	





	


\section{Implementaciones y demostraciones de correctitud}

	\subsection{Implementaciones}
		
		\subsubsection{hayBallotage}
		% Para hacer que quede todo en una misma linea, se puede usar minipage
			\begin{minipage}[t]{\textwidth}
				\begin{lstlisting}[label=code:for]
					res := true
					trans := 0
					primero := 0
					segundo := 0
					i := 0
					suma := 0
					while (escrutinio.size() > i) do
						suma:= suma + escrutinio[i]
						i := i + 1
					endwhile
					i := 0
					while (escrutinio.size() > i) do
						escrutinio[i] := (escrutinio[i] * 100)/suma
						i := i + 1
					endwhile
					i := 0
					while (escrutinio.size() > i) do 
						if (segundo < escrutinio[i])
							segundo := escrutinio[i]
						else:
							skip
						endif
						if (primero < segundo)
							trans := primero
							primero := segundo
							segundo := trans
						else:
							skip
						endif
						i := i + 1
					endwhile
					if (primero > 45)
						res := false
					else
						if ((primero > 40) && (primero - segundo >= 10))
							res := false
						else
							skip
						endif
					endif

				\end{lstlisting}
			\end{minipage}

		\subsubsection{hayFraude}
			\begin{minipage}[t]{\textwidth}
				\begin{lstlisting}[label=code:for]
    
                        res := false
						i := 0
						SumaSen := 0
						sumaDip := 0
						sumaPres := 0
						while (escrutinio_Presidente.size() > i) do
							sumaPres := sumaPres + escrutinio_Presidente[i]
							sumaDip := sumaDip + escrutinio_Diputados[i]
							sumaSen := sumaSen + escrutinio_Senadoresl[i]
							i := i + 1
						endwhile
						if (sumaPres = sumaDip && sumaPres = sumaSen) then 
							skip
						else:
							res := true
						endif
				\end{lstlisting}
			\end{minipage}

		\subsubsection{obtenerSenadoresEnProvincia}
			\begin{minipage}[t]{\textwidth}
				\begin{lstlisting}[label=code:for]

						trans := 0
						if (escrutinio[0] > escrutinio[1]):
    							primero := 0
    							segundo := 1
						else:
    							primero := 1
    							segundo := 0
						endif
						i := 2
						while (escrutinio.size() > i) do
							if (escrutinio[segundo] < escrutinio[i])
								segundo := i
							else:
								skip
							endif
							if (escrutinio[primero] < escrutinio[segundo])
								trans := primero
								primero := segundo
								segundo :=  trans
							else:
								skip
							endif
							i := i + 1
						endwhile	
						res := (primero,segundo)
				\end{lstlisting}
			\end{minipage}
	
		\subsubsection{validarListasDiputadosEnProvincia}
			\begin{minipage}[t]{\textwidth}
				\begin{lstlisting}[label=code:for]
					res := true
					i := 0
					while (listas.size() > i) do
						if (listas[i].size() != cant_bancas) 
							res:= false
						else:
							skip
						endif
						i := i + 1
					endwhile
					i := 0
					while (listas.size() > i) do
                        j := 1
						genero := listas[i][0][1]
						while (listas[i].size() > j) do
							if (listas[i][j][1] == genero) 
								res:=false
							else:
								genero := listas[i][j][1]
								j := j + 1
							endif
					    endwhile
						i := i + 1
					endwhile
				\end{lstlisting}
			\end{minipage}
		
	\subsection{Demostraciones de correctitud}
\subsubsection{hayFraude}
\begin{itemize}
        \item e1 : escrutinioPresidente, \: e2 : escrutinioSenadores, \: e3 : escrutinioDiputados
        \item $P_c : res=False \land i=0 \land sumaPres=0 \land sumaDip=0 \land sumaSen=0$
        
	\item $Q_c : \sum\limits_{i=0}^{|e1| - 1} e1[i] = sumaPres \land \sum\limits_{i=0}^{|e2| - 1} e2[i] = sumaSen \land \sum\limits_{i=0}^{|e3| - 1} e3[i] = sumaDip \land \\ 
 (res = False \longleftrightarrow sumaPres = SumaDip \land sumaPres = sumaSen)$
	\item $B : |e1| > i$
	\item $F_v : |e1| - i$
        \item $Post : res = \neg{(sumaDeVotos(e1) = sumaDeVotos(e3) \land sumaDeVotos(e1) = sumaDeVotos(e2))}$
	\item $I : 0 \leq i \leq |e1| \land (\sum\limits_{j=0}^{i - 1} e1[j] = sumaPres \land \sum\limits_{j=0}^{i - 1} e2[j] = sumaSen \land \sum\limits_{j=0}^{i - 1} e3[j] = sumaDip) \land \\
    (sumaPres = sumaDip  \land sumaPres = sumaSen \longleftrightarrow res = False)$
    
\end{itemize}
\vspace{6mm} 
\begin{enumerate}
    \item  Pre $\longrightarrow_{L}$ wp(sumaPres := 0; sumaDip := 0; sumaSen := 0; i := 0; res := False, Pc)
    \vspace{3mm} 

    $wp(sumaDip := 0, \: wp(sumaDip := 0, \: wp(sumaSen := 0, \: wp(i := 0, \: wp(res := False, \: P_c))))$

    \quad $wp(res := False, P_c) \equiv def(False) \land_L  Pc^{res}_{False} \equiv$

    \quad\quad $\equiv True \land_L (False = False \land i = 0 \land sumaPres = 0 \land sumaDip = 0 \land sumaSen = 0)$

    \quad\quad $\equiv i = 0 \land sumaPres = 0 \land sumaDip = 0 \land sumaSen = 0 \equiv p0$
    
    \quad $wp(i := 0, p0) \equiv def(0) \land_L  p0^i_0 \equiv$

    \quad\quad $\equiv True \land_L 0 = 0 \land sumaPres = 0 \land sumaDip = 0 \land sumaSen = 0 $

    \quad\quad $\equiv sumaPres = 0 \land sumaDip = 0 \land sumaSen = 0 \equiv p1$

    \quad $wp(sumaSen := 0, \: p1) \equiv def(0) \land_L p1^{sumaSen}_0$

    \quad\quad $\equiv True \land_L sumaPres = 0 \land sumaDip = 0 \land 0 = 0$

    \quad\quad $ \equiv sumaPres = 0 \land sumaDip = 0 \equiv p2$

    \quad $wp(sumaDip := 0, \: p2) \equiv def(0) \land_L p2^{sumaDip}_0$

    \quad\quad $\equiv True \land_L sumaPres = 0 \land 0 = 0$

    \quad\quad $\equiv sumaPres = 0 \equiv p3$

    \quad $wp(sumaPres := 0, p3) \equiv def(0) \land_L p3^{sumaPres}_0 $

    \quad\quad $\equiv True \land_L 0 = 0$  

    \quad\quad$ \equiv True $ 

\vspace{3mm} 
{Pre $\longrightarrow_{L}$ wp(sumaPres := 0; sumaDip := 0; sumaSen := 0; i :=0; res := False, Pc) \\ es verdadero porque Pre siempre implica a True }
\vspace{5mm} 

\item $Q_c \longrightarrow_{L} \text{wp}(\text{If}(\ldots), \: Post)$ 
\vspace{3mm}

\quad $wp(\text{If}(\dots), Post) \equiv$

\quad\quad $B : sumaPres = sumaDip \land sumaPres = sumaSen$

\quad\quad $\equiv def(B) \land_L ((B \land \textcolor{red}{wp}(Skip, \: Post) \lor (\neg B \land \textcolor{blue}{wp}(res := True , \: Post))))$

\quad\quad $\equiv True \land_L ((B \land \textcolor{red}{wp}(Skip, \: Post) \lor (\neg B \land \textcolor{blue}{wp}(res := True, \: Post))))$

\quad\quad\quad $\textcolor{red}{wp}(Skip, \: Post) \equiv Post$

\quad\quad\quad $\textcolor{blue}{wp}(res := True, \: Post) \equiv$

\quad\quad\quad\quad $\equiv def(True) \land_v Post^{res}_{True}$

\quad\quad\quad\quad $\equiv True \land_v Post^{res}_{True}$

\quad\quad\quad\quad $\equiv True = \neg{(sumaDeVotos(e1) = sumaDeVotos(e3) \: \land sumaDeVotos(e1) = sumaDeVotos(e2))}$

\quad\quad\quad\quad $\textcolor{blue}{\equiv} \: True = \neg(\sum\limits_{i=0}^{|e1| - 1} e1[i] = \sum\limits_{i=0}^{|e3| - 1} e3[i] \: \land
 \sum\limits_{i=0}^{|e1| - 1} e1[i] = \sum\limits_{i=0}^{|e2| - 1} e2[i]) \equiv \textcolor{purple}{C}$ 

\quad\quad $\equiv True \land_L ((B \land Post) \lor (\neg B \land \textcolor{blue}{wp}(res := True, \: Post))) \equiv 
((B \land Post) \lor (\neg{B} \land \textcolor{purple}{C})) \equiv$

\quad\quad $\equiv ((sumaPres = sumaDip \land sumaPres = sumaSen \land Post) \: \lor$ 

\quad\quad\quad $(\neg{(sumaPres = sumaDip \land sumaPres = sumaSen) \land \textcolor{purple}{C}}))$

\vspace{1mm} 

\quad $ wp(\text{If}(\dots), Post) \equiv ((sumaPres = sumaDip \land sumaPres = sumaSen \land Post) \: \lor$ 

\quad\quad $(\neg{(sumaPres = sumaDip \land sumaPres = sumaSen) \land \textcolor{purple}{C}}))$

\vspace{3mm} 

Recordatorio de $Q_C$ : $\sum\limits_{i=0}^{|e1| - 1} e1[i] = sumaPres \land \sum\limits_{i=0}^{|e2| - 1} e2[i] = sumaSen \land \sum\limits_{i=0}^{|e3| - 1} e3[i] = sumaDip \land \\ 
 (res = False \longleftrightarrow sumaPres = SumaDip \land sumaPres = sumaSen)$

\vspace{1mm} 

Veamos si $Q_c \longrightarrow_{L} \text{wp}(\text{If}(\ldots), \: Post)$ por partes. Asumimos $Q_c $ verdadero. Vemos que nos basta con probar una de \\ las ramas del wp, que se separa en dos casos: cuando se cumple B o $\neg{B}$. Sabemos que \\ $(res = False \longleftrightarrow sumaPres = SumaDip \land sumaPres = sumaSen)$ entonces sumaPres = SumaDip, \\ sumaPres = sumaSen y res = False. \\ Si probamos $Q_C \longrightarrow_{L} Post$, queda probada la implicación porque es verdadera la rama B del wp. \\
Sabemos que Post $\equiv res = \neg(\sum\limits_{i=0}^{|e1| - 1} e1[i] = \sum\limits_{i=0}^{|e3| - 1} e3[i] \: \land
 \sum\limits_{i=0}^{|e1| - 1} e1[i] = \sum\limits_{i=0}^{|e2| - 1} e2[i])$

 
Las sumatorias, sabemos por $Q_c$, que son iguales a la variable con el resultado de la suma, por ejemplo: $\sum\limits_{i=0}^{|e1| - 1} e1[i]$ = sumaPres y que sumaPres = sumaDip y sumaPres = sumaSen. Ademas, sabemos que res = False. Entonces: \\ 

\begin{center}
    $Post \equiv
res = \neg(\sum\limits_{i=0}^{|e1| - 1} e1[i] = \sum\limits_{i=0}^{|e3| - 1} e3[i] \: \land \sum\limits_{i=0}^{|e1| - 1} e1[i] = \sum\limits_{i=0}^{|e2| - 1} e2[i])
$
    $Post \equiv False = \neg{(sumaPres = sumaDip \land sumaPres = sumaSen)}$
    
    $Post \equiv False = \neg{(True \land True)}$

    $Post \equiv False = False$

    $Post \equiv True$

\end{center}

Entonces $Q_c \longrightarrow_{L} \text{wp}(\text{If}(\ldots), \: Post)$
\item $P_c \longrightarrow_{L}$ wp(\text{While}(\ldots), Qc) mediante el teorema del invariante.

e1 : escrutinioPresidente, \: e2 : escrutinioSenadores, \: e3 : escrutinioDiputados

\begin{enumerate}
    \item $P_c \longrightarrow I$ \\
    $P_c : res=False \land i=0 \land sumaPres=0 \land sumaDip=0 \land sumaSen=0$ \\
    $I : 0 \leq i \leq |e1| \land (\sum\limits_{j=0}^{i - 1} e1[j] = sumaPres \land \sum\limits_{j=0}^{i - 1} e2[j] = sumaSen \land \sum\limits_{j=0}^{i - 1} e3[j] = sumaDip) \land \\
    (sumaPres = sumaDip  \land sumaPres = sumaSen \longleftrightarrow res = False) 
    $
    \begin{itemize}
        \item Como i = 0, se cumple que 0 $\leq$ i $\leq |e1|$
        \item Como i = 0, las sumatorias suman 0 (suma desde un limite a otro menor) y tenemos:
        \begin{center}
            0 = sumaPres $\land$ 0 = sumaSen $\land$ 0 = sumaDip 
        \end{center}
        \item Como las tres variables, sumaPres, sumaDip y sumaSen son iguales a 0, sumaPres = sumaDip y sumaPres = sumaSen. Res tambien es False, por lo que es verdadera la ultima condicion del invariante.
    \end{itemize}
    Entonces $P_c \longrightarrow I$
    
    
    \item $I \land \neg{B} \longrightarrow Q_c$ \\
    
    $I : 0 \leq i \leq |e1| \land (\sum\limits_{j=0}^{i - 1} e1[j] = sumaPres \land \sum\limits_{j=0}^{i - 1} e2[j] = sumaSen \land
    \sum\limits_{j=0}^{i - 1} e3[j] = sumaDip) \land \\
    (sumaPres = sumaDip  \land sumaPres = sumaSen \longleftrightarrow res = False)$ \\
    
    $\neg{B} : \neg{(|e1| > i)} \equiv |e1| \leq i $ \\
    $Q_c :  \sum\limits_{i=0}^{|e1| - 1} e1[i] = sumaPres \land \sum\limits_{i=0}^{|e2| - 1} e2[i] = sumaSen \land \sum\limits_{i=0}^{|e3| - 1} e3[i] = sumaDip \land \\ 
 (res = False \longleftrightarrow sumaPres = SumaDip \land sumaPres = sumaSen)$
 \begin{itemize}
     \item Como i $\leq |e1|$ y tambien $|e1| \leq i$ entonces i = $|e1|$. Al reemplazar i por $|e1|$ en las sumatorias de I, se cumplen las igualdades de las sumatorias de $Q_c$.
     \item Del I, tenemos inmediatamente la implicacion de la doble igualdad de $Q_c$.
 \end{itemize}
 Entonces $I \land \neg{B} \longrightarrow Q_c$
    
\item $I \land fv \leq 0 \longrightarrow \neg{B}$ \\
$I : 0 \leq i \leq |e1| \land (\sum\limits_{j=0}^{i - 1} e1[j] = sumaPres \land \sum\limits_{j=0}^{i - 1} e2[j] = sumaSen \land \sum\limits_{j=0}^{i - 1} e3[j] = sumaDip) \land \\
    (sumaPres = sumaDip  \land sumaPres = sumaSen \longleftrightarrow res = False)$ \\
    
$fv \leq 0 : |e1| - i  \leq 0$ \\
$\neg{B} : |e1| \leq i$
\begin{itemize}
    \item De fv $\leq 0: |e1| - i \leq 0 \equiv |e1| \leq i,$ 
    Que es exactamente $\neg{B}$
\end{itemize}

Entonces $I \land fv \leq 0 \longrightarrow \neg{B}$

\item $I \land B \longrightarrow wp(sumaPres := sumaPres + e1[i](\ldots);\: i \: := \: i + 1, \: I)$\\
$I : 0 \leq i \leq |e1| \land (\sum\limits_{j=0}^{i - 1} e1[j] = sumaPres \land \sum\limits_{j=0}^{i - 1} e2[j] = sumaSen \land \sum\limits_{j=0}^{i - 1} e3[j] = sumaDip) \land \\
    (sumaPres = sumaDip  \land sumaPres = sumaSen \longleftrightarrow res = False)$ \\
$B : |e1| > i$ \\
$wp(sumaPres := sumaPres + e1[i](\ldots);\: i \: := \: i + 1, \: I) \equiv$ 

\quad\quad \textcolor{cyan}{wp}(sumaPres := sumaPres + e1[i], \textcolor{blue}{wp}(sumaDip := sumaDip + e3[i],

\quad\quad \textcolor{green}{wp}(sumaSen := sumaSen + e2[i], \textcolor{red}{wp}(i := i + 1, I))))

\quad\quad\quad \textcolor{red}{wp}(i := i + 1, I) $\equiv def(i+1) \: \land_{L} I^i_{i+1} \equiv$

\quad\quad\quad $\equiv True \: \land_{L} I^i_{i+1}$

\quad\quad\quad $\equiv 0 \leq i + 1 \leq |e1| \land (\sum\limits_{j=0}^{i} e1[j] = sumaPres \land \sum\limits_{j=0}^{i} e2[j] = sumaSen \land \sum\limits_{j=0}^{i} e3[j] = sumaDip) \land \\
    (sumaPres = sumaDip  \land sumaPres = sumaSen \longleftrightarrow res = False) \equiv \textcolor{red}{wp1} $\\

\quad\quad\quad \textcolor{green}{wp}(sumaSen := sumaSen + e2[i], \textcolor{red}{wp1}) $\equiv def(sumaSen +e2[i]) \land_{L} \textcolor{red}{wp1}^{sumaSen}_{sumaSen + e2[i]}$

\quad\quad\quad $\equiv True \: \land_{L} \textcolor{red}{wp1}^{sumaSen}_{sumaSen + e2[i]}$

\quad\quad\quad $\equiv 0 \leq i + 1 \leq |e1| \land (\sum\limits_{j=0}^{i} e1[j] = sumaPres \land \sum\limits_{j=0}^{i} e2[j] = sumaSen + e2[i] \land \sum\limits_{j=0}^{i} e3[j] = sumaDip) \land \\
    (sumaPres = sumaDip  \land sumaPres = sumaSen + e2[i] \longleftrightarrow res = False) \equiv \textcolor{green}{wp2} $

\quad\quad\quad \textcolor{blue}{wp}(sumaDip := sumaDip + e3[i], \textcolor{green}{wp2}) $\equiv def(sumaDip +e3[i]) \land_{L} \textcolor{green}{wp2}^{sumaDip}_{sumaDip + e3[i]}$

\quad\quad\quad $\equiv True \: \land_{L} \textcolor{green}{wp2}^{sumaDip}_{sumaDip + e3[i]}$

\quad\quad\quad $\equiv 0 \leq i + 1 \leq |e1| \land (\sum\limits_{j=0}^{i} e1[j] = sumaPres \land \sum\limits_{j=0}^{i} e2[j] = sumaSen + e2[i] \land \sum\limits_{j=0}^{i} e3[j] = sumaDip + e3[i]) \land \\
    (sumaPres = sumaDip + e3[i]  \land sumaPres = sumaSen + e2[i] \longleftrightarrow res = False) \equiv \textcolor{blue}{wp3}$

\quad\quad\quad \textcolor{cyan}{wp}(sumaPres := sumaPres + e1[i], \textcolor{blue}{wp3}) $\equiv def(sumaDip +e3[i]) \land_{L} \textcolor{blue}{wp3}^{sumaPres}_{sumaPres + e1[i]}$

\quad\quad\quad $\equiv True \: \land_{L} \textcolor{blue}{wp3}^{sumaPres}_{sumaPres + e1[i]}$

\quad\quad\quad $\equiv 0 \leq i + 1 \leq |e1| \land (\sum\limits_{j=0}^{i} e1[j] = sumaPres + e1[i] \land \sum\limits_{j=0}^{i} e2[j] = sumaSen + e2[i] \land \sum\limits_{j=0}^{i} e3[j] = sumaDip + e3[i]) \land \\
    (sumaPres + e1[i] = sumaDip + e3[i]  \land sumaPres + e1[i] = sumaSen + e2[i] \longleftrightarrow res = False) \equiv \textcolor{cyan}{wp4}$ \\


Veamos si I $\land \: B \longrightarrow$ \textcolor{cyan}{wp4}
\begin{itemize}
    \item De 0 $\leq i \leq |e1|$ y $|e1| > i$ tenemos que $0 \leq i < |e1|$ que implica a $0 \leq i + 1 \leq |e1|$ para todos los valores de i.
    \item Del I, tenemos res = False, por lo que es verdadero tambien (sumaPres + e1[i] = sumaDip + e3[i]  $\land$ sumaPres + e1[i] = sumaSen + e2[i] $\longleftrightarrow$ res = False)
    \item Para probar que las 3 sumatorias son iguales a su respectiva variable más el termino actual del escrutinio correspondiente en i, nos basta con probarla de forma general o probar 1, ya que las otras dos son analogas.
    \begin{center}
        $\sum\limits_{j=0}^{i} e1[j] = sumaPres + e1[i]$
        $\sum\limits_{j=0}^{i} e1[j] = sumaPres + e1[i] \equiv e1[0] + e1[1] + ... + e1[i-1] + e1[i] = sumaPres + e1[i] $ \\
        Restamos el termino e1[i] de ambos lados y nos queda: \\
        
        $e1[0] + e1[1] + ... + e1[i-1] = sumaPres$\\

        Del I, sabemos que $\sum\limits_{j=0}^{i - 1} e1[j] = sumaPres$, que podemos descomponer y ver que vale: \\
        $e1[0] + e1[1] + ... + e1[i-1] = sumaPres$ \\
        ¡Son identicos! Así que el I me prueba las tres sumatorias del wp.
    \end{center}
\end{itemize}
Entonces $I \land B \longrightarrow wp(sumaPres := sumaPres + e1[i](\ldots);\: i \: := \: i + 1, \: I)$

\item $I \land B \land v_0 = |e1| - i \longrightarrow wp(sumaPres := sumaPres + e1[i](\ldots);\: i \: := \: i + 1, \: |s| - i < v_0)$\\
$I : 0 \leq i \leq |e1| \land (\sum\limits_{j=0}^{i - 1} e1[j] = sumaPres \land \sum\limits_{j=0}^{i - 1} e2[j] = sumaSen \land \sum\limits_{j=0}^{i - 1} e3[j] = sumaDip) \land \\
    (sumaPres = sumaDip  \land sumaPres = sumaSen \longleftrightarrow res = False)$ \\
$B : |e1| > i$ \\
$v_0 = |e1| - i$ \\
$wp(sumaPres := sumaPres + e1[i](\ldots);\: i \: := \: i + 1, \: |e1| - i < v_0) \equiv$

\quad\quad \textcolor{cyan}{wp}(sumaPres := sumaPres + e1[i], \textcolor{blue}{wp}(sumaDip := sumaDip + e3[i],

\quad\quad \textcolor{green}{wp}(sumaSen := sumaSen + e2[i], \textcolor{red}{wp}(i := i + 1, $|e1| - i < v_0$))))

\quad\quad\quad \textcolor{red}{wp}(i := i + 1, $|e1| - i < v_0$) $\equiv def(i+1) \: \land_{L} (|e1| - i < v_0)^i_{i+1} \equiv$

\quad\quad\quad $\equiv True \land (|e1| - i < v_0)^i_{i+1}$

\quad\quad\quad $\equiv |e1| - (i+1) < v_0$

\quad\quad\quad $\equiv |e1| - i - 1 < v_0 \equiv \textcolor{red}{wp1}$

\quad\quad\quad \textcolor{green}{wp}(sumaSen := sumaSen + e2[i], $\textcolor{red}{wp1}$) $\equiv def(sumaSen + e2[i]) \: \land_{L} \textcolor{red}{wp1}^{sumaSen}_{sumaSen + e2[i]} \equiv$

\quad\quad\quad $\equiv True \land_L \textcolor{red}{wp1}^{sumaSen}_{sumaSen + e2[i]}$

\quad\quad\quad $\equiv |e1| - i - 1 < v_0 \equiv \textcolor{green}{wp2}$

\quad\quad\quad \textcolor{blue}{wp}(sumaDip := sumaDip + e3[i], $\textcolor{green}{wp2}$) $\equiv def(sumaDip + e3[i]) \: \land_{L} \textcolor{green}{wp2}^{sumaDip}_{sumaDip + e3[i]} \equiv$

\quad\quad\quad $\equiv True \land_L \textcolor{green}{wp2}^{sumaDip}_{sumaDip + e3[i]}$

\quad\quad\quad $\equiv |e1| - i - 1 < v_0 \equiv \textcolor{blue}{wp3}$

\quad\quad\quad \textcolor{cyan}{wp}(sumaPres := sumaPres + e1[i], $\textcolor{blue}{wp3}$) $\equiv def(sumaPres + e1[i]) \: \land_{L} \textcolor{blue}{wp3}^{sumaPres}_{sumaPres + e1[i]} \equiv$

\quad\quad\quad $\equiv True \land_L \textcolor{blue}{wp3}^{sumaPres}_{sumaPres + e1[i]}$

\quad\quad\quad $\equiv |e1| - i - 1 < v_0 \equiv \textcolor{cyan}{wp4}$

Veamos si I $\land \: B \land v_0 = |e1| - i \longrightarrow \textcolor{cyan}{wp4}$
\begin{itemize}
    \item Queremos probar que $|e1| - i - 1 < v_0$. Sabemos que $v_0 = |e1| - i$. Esta igualdad nos dice que $v_0 = |e1| - i$ y por lo tanto mayor a todas las expresiones menores a $|e1| - i$, como por ejemplo $|e1| - i - 1$, que justamente es la expresion de $\textcolor{cyan}{wp4}$.
\end{itemize}
Entonces $I \land B \land v_0 = |e1| - i \longrightarrow wp(sumaPres := sumaPres + e1[i](\ldots);\: i \: := \: i + 1, \: |s| - i < v_0)$
\end{enumerate}
\end{enumerate}
Como probamos:
\begin{itemize}
    \item Pre $\longrightarrow$ wp(codigo previo al ciclo, $P_c$)
    \item $P_c$ $\longrightarrow$ wp(ciclo(por teorema del invariante), $Q_c$)
    \item $Q_c$ $\longrightarrow$ wp(codigo posterior al ciclo, Post)
\end{itemize}
Al probar estas tres cosas, por corolario de monotonía sabemos que Pre $\longrightarrow$ wp(programa completo, Post)
y, por lo tanto, el programa es correcto con respecto a la especificación.



\subsubsection{obtenerSenadoresProvincia}

\begin{itemize}
	\item  S= escrutinio, 1º = primero, 2º = segundo, tº =trans
      \item $P_c = \{ tº=0 \land i=2 \land (1º=0 \land 2° = 1) \lor ( = 1 \land 2°=0) \}$
\item $Q_c = (\forall j:\ent)( 0\leq j < |s| \longrightarrow_{L} ( s[j]\leq s[2°] \longrightarrow s[2°]<s[1°))$
	\item $B = |S| > i$ 
	\item $F_v = |S| - i$
	\item $I = \{0 \leq i \leq |s| \land_{L} (\forall j:\ent)(0 \leq j < i \longrightarrow_{L} S[1º]>S[j]) \land (\forall j:\ent)(0 \leq j < i \land j \neq 1º \longrightarrow_{L} S[1º]>S[j])$
        \item $Q_c = \{ \}$
\end{itemize}
\begin{enumerate}
    \item $P_c \longrightarrow I$ \\
    $P_c : t°=0 \land i=2 \land (1°=0 \land 2° = 1) \lor ( = 1 \land segundo=0)$ \\
    $I : 0 \leq i \leq |s| \land_{L} (\forall j:\ent)(0 \leq j < i \longrightarrow_{L} S[1º]>S[j]) \land (\forall j:\ent)(0 \leq j < i \land j \neq 1º \longrightarrow_{L} S[1º]>S[j])$
    $Q_c : 0\leq j <|S| $
    \begin{itemize}
        \item Como i = 2, se cumple que 0 $\leq$ i $\leq |s|$
            \item Procedemos a observar ambos casos de 1º y 2º (tº no nos importa debido a que no se lo menciona en el I). 
                 \\ Caso 1: 1°=0 $\land$ 2°=1 $\land$ i=0
                    \begin{center}
                        $(\forall j:\ent)(0 \leq j < 0 \longrightarrow_{L} S[0]>S[j]) \land (\forall j:\ent)(0 \leq j < 0 \land j \neq 0 \longrightarrow_{L} S[1]>S[j]) \equiv$
                    \\  $(\forall j:\ent)(False \longrightarrow_{L} S[0]>S[j]) \land (\forall j:\ent)(False \land j \neq 0 \longrightarrow_{L} S[1]>S[j]) \equiv$
                    \\ $True \land True$ \; \text{(por tabla de verdad de la implicacion)}
                    \end{center}
                  Caso 2: 1°=1 $\land$ 2°=0 $\land$ i=0
                    \begin{center}
                        $(\forall j:\ent)(0 \leq j < 0 \longrightarrow_{L} S[1]>S[j]) \land (\forall j:\ent)(0 \leq j < 0 \land j \neq 1 \longrightarrow_{L} S[0]>S[j]) \equiv $
                    \\  $(\forall j:\ent)(False \longrightarrow_{L} S[0]>S[j]) \land (\forall j:\ent)(False \land j \neq 1 \longrightarrow_{L} S[1]>S[j]) \equiv $
                    \\  $True \land True \;$\text{(por tabla de verdad de la implicacion)}
                    \end{center}
              \text{Luego como se cumple en ambos casos vale entonces $P_c \longrightarrow I$.}
    \end{itemize}
    \item $I \land \neg B \longrightarrow Q_c$
    $I : 0 \leq i \leq |s| \land_{L} (\forall j:\ent)(0 \leq j < i \longrightarrow_{L} S[1º]>S[j]) \land (\forall j:\ent)(0 \leq j < i \land j \neq 1º \longrightarrow_{L} S[1º]>S[j])$
    $B : |S| > i$
    
    

\end{enumerate}
\end{document}


