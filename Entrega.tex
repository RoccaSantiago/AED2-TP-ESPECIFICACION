\documentclass [10 pt , a4paper ]{ article }
\usepackage { caratula } 
\titulo { Descripcion del tp }
\subtitulo { Subtitulo del tp }
\fecha {\ today }
\materia { Materia de la carrera }
\grupo { Grupo 42}
\integrante { Lopez, Gonzalo }{1017/22}{ gonzalo.esloga.uba@gmail.com }
\integrante { Gomez, Abril  }{574/20}{goskema@gmail.com }
\integrante { Fisz, Maximiliano }{002/01}{ email1@dominio . com }
\integrante { Rocca, Santiago }{152/23}{ santiagorocca17@gmail.com }
\begin{tad}{TP}

    \tadObservadores
    \tadOperacion{entregado}{TP/tp,ID/id}{bool}{existe?(id, tp)}
    
    
    \tadAxiomas
    \tadAxioma{entregado(tp, id)}{false}
    \end{tad}
    
    \section{Módulo TP}
    
    \begin{Interfaz}
    
      \textbf{se explica con}: \tadNombre{TP}
    
      \textbf{géneros}: \TipoVariable{TP}.
    
      \Titulo{Operaciones básicas de TP}
    
      \InterfazFuncion{Generador\_1}{}{TP}%
      {$res$ $\igobs$ gen}%
      [$\Theta(1)$]
      [genera un nuevo tp.]
    
      \InterfazFuncion{Observador\_1}{\Inout{tp}{TP}, \In{n}{Nat}}{string}
      [$n < 10$]
      {$res$ $\igobs$ numeroATexto(obs1(tp), n)}
      [$E = MC^2$]
      [El mejor observador del universo]
      [¿Esto con qué se come?]
    
    \end{Interfaz}
    
    \end{document}